\documentclass[a4paper,11pt]{scrartcl}

\usepackage{amsmath,amssymb}
\usepackage[utf8]{inputenc}

\title{Derivatives of the likelihood for a beta-psychometric function}
\author{Ingo Fründ}
\date{\today}

\def\rd{\mathrm{d}}
\def\p{\partial}
\def\ddlnu{\frac{\p^2\ell_i}{\p\nu^2}}
\def\ddlf{\frac{\p^2\ell_i}{\p f_i^2}}
\def\ddlfnu{\frac{\p^2\ell_i}{\p f_i\p\nu}}
\def\dfdt{\frac{\p f_i}{\p\theta}}

\begin{document}

\maketitle
% Your text here

\section{Log likelihood}

The log likelihood for a beta-psychometric function can be written as
%
$$
\ell = \sum_{i=1}^m \ell_i,
$$
%
where $m$ is the number of blocks in the experiment and
%
$$
\ell_i = \log\Gamma(\nu n_i) - \log\Gamma( f_i\nu n_i ) - \log\Gamma( (1-f_i) \nu n_i ) + ( f_i \nu n_i - 1 ) \log p_i + ( (1-f_i) \nu n_i - 1 ) \log (1-p_i).
$$
%
Here, $f_i = \gamma + (1-\gamma-\lambda) F(g(x_i,\alpha,\beta))$, i.e. the predicted value of the psychometric function, and
$p_i = k_i/n_i$, i.e. the observed probability of a correct response.

\section{Derivatives of the log likelihood function}

Derivatives of the log likelihood function involve the $\psi$ function,
%
$$
\psi = \frac{\rd}{\rd x} \log\Gamma.
$$
%
They read:
%
$$
\frac{\p \ell_i}{\p\nu} = n_i\psi(\nu n_i) - f_i n_i \psi ( f_i\nu n_i) - (1-f_i) n_i \psi ( (1-f_i) \nu n_i) + f_i n_i \log(p_i) + (1-f_i) n_i \log(1-p_i),
$$
%
and
%
$$
\frac{\p \ell_i}{\p f_i} = \nu n_i \Big( \log\frac{p_i}{1-p_i} + \psi ( (1-f_i) \nu n_i) - \psi ( f_i \nu n_i) \Big).
$$
%
In order to obtain derivatives of the log likelihood with respect to parameters $\alpha,\beta,\gamma,\delta$, we use the definition of $f_i$ to arrive at
%
$$
\frac{\p \ell_i}{\p \theta} = \frac{\p\ell_i}{\p f_i} \frac{\p f_i}{\p \theta}, \quad \theta\in\{\alpha,\beta,\gamma,\delta\}.
$$
%
These are given by:
%
\begin{align*}
    \frac{\p f_i}{\p\alpha} &= (1-\gamma-\lambda) F' ( g(x_i,\alpha,\beta) ) \frac{\p g}{\p \alpha} ( x_i, \alpha, \beta ),\\
    \frac{\p f_i}{\p\beta} &= (1-\gamma-\lambda) F' ( g(x_i,\alpha,\beta) ) \frac{\p g}{\p \beta} ( x_i, \alpha, \beta ),\\
    \frac{\p f_i}{\p\lambda} &= -F(g(x_i,\alpha,\beta)),\\
    \frac{\p f_i}{\p \gamma} &= 1-F(g(x_i,\alpha,\beta)).
\end{align*}
%

\section{Second derivatives of the log likelihood function}

The second derivatives of the log likelihood function involve derivatives of the $\psi$ function.
These are called the trigamma function and are notated by $\psi'$.
%
$$
\ddlnu = \psi' (\nu n_i) n_i^2 - f_i^2 n_i^2\psi' (f_i\nu n_i) - (1-f_i)^2n_i^2 \psi' ( (1-f_i)\nu n),
$$
%
and
%
$$
\ddlf = - (\nu n)^2 \Big(\psi' ( f_i \nu n) + \psi'( (1-f_i)\nu n) \Big).
$$
Finally, we have
%
$$
\ddlfnu = - n_i \psi ( f_i \nu n_i ) - f_i n_i^2\nu \psi' ( f_i\nu n_i ) + n_i\psi( (1-f_i)\nu n_i) + (1-f)n_i^2\nu\psi'( (1-f_i)\nu n_i) + n_i \log\frac{p_i}{1-p_i}.
$$
%

We can use these identities to derive expressions for derivatives of the log likelihood function with respect to parameters of the psychometric function:
%
$$
\frac{\p^2 \ell_i}{\p \lambda^2} = \frac{\p^2\ell_i}{\p \gamma^2} = 0,
$$
%
$$
\frac{\p^2\ell_i}{\p\alpha^2} = (1-\gamma-\lambda) \ddlf \cdot \Big( F''(g(x_i,\alpha,\beta))\frac{\p g}{\p\alpha}(x_i,\alpha,\beta) + F'(g(x_i,\alpha,\beta))\frac{\p^2g}{\p\alpha^2}(x_i,\alpha,\beta)\Big),
$$
%
$$
\frac{\p^2\ell_i}{\p\beta^2} = (1-\gamma-\lambda) \ddlf \cdot \Big( F''(g(x_i,\alpha,\beta))\frac{\p g}{\p\beta}(x_i,\alpha,\beta) + F'(g(x_i,\alpha,\beta))\frac{\p^2g}{\p\beta^2}(x_i,\alpha,\beta)\Big).
$$
%

The crossterms are
%
\begin{align*}
\frac{\p^2\ell_i}{\p\alpha\p\beta} &= (1-\gamma-\lambda)\ddlf\cdot \Big( F''(g(x_i,\alpha,\beta))\frac{\p g}{\p\alpha}\frac{\p g}{\p\beta} + F'(g(x_i,\alpha,\beta))\frac{\p^2}{\p\alpha\p\beta}\Big), \\
\frac{\p^2\ell_i}{\p\alpha\p\gamma} &= - \ddlf F'(g(x_i,\alpha,\beta)) \frac{\p g}{\p\alpha},
\end{align*}
%
and an analogous expression for $\frac{\p^2\ell_i}{\p\beta\p\gamma}$.
Further
%
$$
\frac{\p^2\ell_i}{\p\alpha\p\gamma} = \ddlf (1-F'(g(x_i,\alpha,\beta)) \frac{\p g}{\p\alpha}),
$$
%
with an analogous expression for $\frac{\p^2\ell_i}{\p\beta\p\lambda}$.

We now come the the final crossterm
%
\begin{align*}
\frac{\p^2\ell_i}{\p\theta\p\nu} &= 
    - n_i\dfdt \Big ( \psi (f_i \nu n_i) + f_i\nu n_i \psi'(f_i\nu n_i)\Big)\\
    &\quad+ n_i\dfdt \Big ( \psi ( (1-f_i)\nu n_i) + (1-f_i)\nu n_i\psi'( (1-f_i)\nu n_i) \Big)\\
    &\quad+ n_i\dfdt \log\frac{p_i}{1-p_i},
\end{align*}
%
where $\theta \in\{\alpha,\beta,\lambda,\gamma\}$.

\end{document}
